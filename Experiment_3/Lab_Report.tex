\documentclass[11pt]{article}
\usepackage[margin=1in]{geometry}
\usepackage{enumitem}
\usepackage{hyperref}
\usepackage{graphicx}
\usepackage{array}
\usepackage{multicol}
\usepackage{longtable}
\usepackage{titlesec}

\begin{document}

%==================================================
\begin{center}
    \large \textbf{Sri Sivasubramaniya Nadar College of Engineering, Chennai} \\
    (An Autonomous Institution Affiliated to Anna University) \\
    \vspace{0.3cm}
\end{center}

\begin{table}[!h]
\renewcommand{\arraystretch}{1.5}
\resizebox{\textwidth}{!}{%
\begin{tabular}{|l|cll|}
\hline
Degree \& Branch & \multicolumn{1}{c|}{B.E. Computer Science \& Engineering} & Semester & VI \\ \hline
Subject Code \& Name & \multicolumn{3}{c|}{UCS2612 -- Machine Learning Algorithms Laboratory} \\ \hline
Academic Year & 2025--2026 (Even) & Batch & 2023--2027 \\ \hline
Due Date & \multicolumn{3}{c|}{\textbf{}} \\ \hline
\end{tabular}
}
\end{table}

\begin{center}
\textbf{Experiment 3: Regression Analysis using Linear and Regularized Models}
\end{center}

%==================================================
\section*{Objective}
To implement linear and regularized regression models for predicting a continuous target variable, evaluate their performance using multiple metrics, visualize model behavior, and analyze overfitting, underfitting, and bias--variance characteristics.

%==================================================
\section*{Dataset}
A real-world regression dataset containing numerical and categorical features related to loan applications is used.  
The target variable is the \textbf{loan amount sanctioned}.

Dataset reference:
\begin{itemize}
    \item Kaggle: \href{https://www.kaggle.com/datasets/phileinsophos/predict-loan-amount-data}{Predict Loan Amount Data}
\end{itemize}

%==================================================
\section*{Brief Theory (For Lab Understanding)}

\subsection*{Linear Regression}
Linear Regression models the relationship between input features and a continuous target variable.
It is simple, interpretable, and serves as a baseline regression model.

\subsection*{Regularized Regression Models}
Regularization techniques are used to control model complexity:
\begin{itemize}
    \item Ridge Regression reduces coefficient magnitudes
    \item Lasso Regression performs feature selection
    \item Elastic Net combines Ridge and Lasso behavior
\end{itemize}

Regularization helps improve generalization and reduce overfitting.

%==================================================
\section*{Task Description}
Students must:
\begin{itemize}
    \item Implement Linear, Ridge, Lasso, and Elastic Net regression models
    \item Tune regularization hyperparameters using Grid Search or Randomized Search
    \item Visualize regression results and errors
    \item Analyze overfitting, underfitting, and bias--variance trade-off
\end{itemize}

%==================================================
\section*{Implementation Steps}
\begin{enumerate}[label=\arabic*.]
    \item Load the dataset
    \item Perform data preprocessing:
    \begin{itemize}
        \item Handle missing values
        \item Encode categorical variables
        \item Standardize numerical features
    \end{itemize}
    \item Perform Exploratory Data Analysis (EDA)
    \item Visualize feature distributions and target distribution
    \item Split the dataset into training and testing sets
    \item Train baseline Linear Regression
    \item Train Ridge, Lasso, and Elastic Net models
    \item Perform hyperparameter tuning using 5-Fold Cross-Validation
    \item Evaluate all models using regression metrics
\end{enumerate}

%==================================================
\section*{Required Visualizations (During Coding)}
Students must generate and include:
\begin{itemize}
    \item Target variable distribution plot
    \item Feature vs. target scatter plots
    \item Predicted vs. actual values plot
    \item Residual plot
    \item Training error vs. validation error plot
    \item Coefficient comparison bar plot
\end{itemize}

%==================================================
\section*{Performance Metrics to be Reported}
\begin{itemize}
    \item Mean Absolute Error (MAE)
    \item Mean Squared Error (MSE)
    \item Root Mean Squared Error (RMSE)
    \item $R^2$ Score
    \item Training Time
\end{itemize}

%==================================================
\section*{Hyperparameter Search Space}
\begin{itemize}
    \item Ridge: $\alpha \in \{0.01, 0.1, 1, 10, 100\}$
    \item Lasso: $\alpha \in \{0.001, 0.01, 0.1, 1, 10\}$
    \item Elastic Net:
    \begin{itemize}
        \item $\alpha \in \{0.01, 0.1, 1, 10\}$
        \item $l1\_ratio \in \{0.2, 0.5, 0.8\}$
    \end{itemize}
\end{itemize}

%==================================================
\section*{Hyperparameter Tuning Results}
\begin{table}[h!]
\centering
\renewcommand{\arraystretch}{1.3}
\caption{Hyperparameter Tuning Summary}
\begin{tabular}{|l|c|c|c|}
\hline
Model & Search Method & Best Parameters & Best CV $R^2$ \\ \hline
Ridge Regression & Grid / Random &  &  \\
Lasso Regression & Grid / Random &  &  \\
Elastic Net Regression & Grid / Random &  &  \\ \hline
\end{tabular}
\end{table}

%==================================================
\section*{Cross-Validation Performance (K = 5)}
\begin{table}[h!]
\centering
\renewcommand{\arraystretch}{1.3}
\caption{Cross-Validation Performance}
\begin{tabular}{|l|c|c|c|c|}
\hline
Model & MAE & MSE & RMSE & $R^2$ \\ \hline
Linear Regression &  &  &  &  \\
Ridge Regression &  &  &  &  \\
Lasso Regression &  &  &  &  \\
Elastic Net Regression &  &  &  &  \\ \hline
\end{tabular}
\end{table}

%==================================================
\section*{Test Set Performance Comparison}
\begin{table}[h!]
\centering
\renewcommand{\arraystretch}{1.3}
\caption{Test Set Performance}
\begin{tabular}{|l|c|c|c|c|}
\hline
Model & MAE & MSE & RMSE & $R^2$ \\ \hline
Linear Regression &  &  &  &  \\
Ridge Regression &  &  &  &  \\
Lasso Regression &  &  &  &  \\
Elastic Net Regression &  &  &  &  \\ \hline
\end{tabular}
\end{table}

%==================================================
\section*{Effect of Regularization on Coefficients}
\begin{table}[h!]
\centering
\renewcommand{\arraystretch}{1.3}
\caption{Coefficient Comparison}
\begin{tabular}{|l|c|c|c|c|}
\hline
Feature & Linear & Ridge & Lasso & Elastic Net \\ \hline
Feature 1 &  &  &  &  \\
Feature 2 &  &  &  &  \\
Feature 3 &  &  &  &  \\ \hline
\end{tabular}
\end{table}

%==================================================
\section*{Overfitting and Underfitting Analysis}
Students must discuss:
\begin{itemize}
    \item Difference between training and validation errors
    \item Effect of regularization strength
    \item Improvement in generalization after tuning
\end{itemize}

%==================================================
\section*{Bias--Variance Analysis}
Students must comment on:
\begin{itemize}
    \item Bias behavior of Linear Regression
    \item Variance reduction using Ridge and Elastic Net
    \item Feature sparsity effect in Lasso
\end{itemize}

%==================================================
\section*{Conclusion}
Summarize the performance of all regression models, justify the choice of optimal hyperparameters, and discuss the trade-off between accuracy and model complexity.

%==================================================
\section*{Report Format (Mandatory)}
\begin{enumerate}
    \item Aim and Objective
    \item Dataset Description
    \item Preprocessing Steps
    \item Implementation Details
    \item Visualizations
    \item Performance Tables
    \item Overfitting and Underfitting Analysis
    \item Bias--Variance Analysis
    \item Observations and Conclusion
\end{enumerate}

%==================================================
\section*{References}
\begin{itemize}
    \item \href{https://scikit-learn.org/stable/modules/linear_model.html}{Scikit-learn: Linear Models}
    \item \href{https://scikit-learn.org/stable/modules/grid_search.html}{Scikit-learn: Hyperparameter Optimization}
    \item \href{https://www.kaggle.com/datasets/phileinsophos/predict-loan-amount-data}{Loan Amount Dataset}
\end{itemize}

\end{document}
